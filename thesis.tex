% NEW!
% package options 
%
% difcul          - Generates DI/FCUL cover pages
% msc - MSc dissertation
% bc, bio   - Thesis specialization (bioinformatica, biologica computacional)
\documentclass[twoside,12pt,difcul,bio,msc]{Classes/FCULthesisPSnPDF}


\usepackage[utf8x]{inputenc}
\usepackage[T1]{fontenc}
\usepackage{multirow}
\usepackage{graphicx}
\usepackage{color,array,colortbl} 
\usepackage{amssymb}
\usepackage{amsmath}
\usepackage{booktabs}
\usepackage{listings}

\usepackage{framed}
\usepackage{soul}

\definecolor{light-gray}{gray}{0.9}
\sethlcolor{light-gray}

\newcommand{\argmax}[1]{\underset{#1}{\operatorname{arg}\,\operatorname{max}}\;}

%input macros (i.e. write your own macros file called preamble.tex)
%\include{Macros/macros}


\begin{document}

\title{Semantic annotation of electronic health records in a multilingual environment}
\author{Luís Campos}
\authemail{luis.filipe.lcampos@gmail.com}
\thesisadvisor{Francisco José Moreira Couto}{1} %1 - male 2 - female
\thesiscoadvisor{Vasco Calais Pedro}{1}
\keywords{Translation, Named-entity Recognition, Parallel Corpus, Radiology, RadLex}
\keywordspt{Tradução, Reconhecimento de	Entidades, Corpus Paralelo, Radiology, RadLex}
\degreedate{2017}
%\thextjuri{Miguel Francisco Almeida Pereira Rocha}
%\thintjuri{Andr\'{e} Os\'{o}rio e Cruz de Azer\^{e}do Falc\~{a}o}{Jo\~{a}o Pedro Guerreiro Neto}

\frontmatter
\maketitle

%set the number of sectioning levels that get number and appear in the contents
\setcounter{secnumdepth}{3}
\setcounter{tocdepth}{3}



\includex{Preface}{resumo}
\includex{Preface}{abstract}
%\includex{Preface}{resumolongo}
\includex{Preface}{acknowledgements}
%\includex{Preface}{dedication}

\tableofcontents
\listoffigures
\listoftables
%\listofalgorithms
%\lstlistoflistings
%\printglossary
%\addcontentsline{toc}{chapter}{Nomenclature}
%\include{Conventions/conventions}

\mainmatter

\chapter{Introduction}
\inputx{Chapters}{1_introduction} % tex file in Chapters/
\chapter{Related Work}
\inputx{Chapters}{2_related_work}
\chapter{Multilingual NER of Radiology Text}
\inputx{Chapters}{3_framework}
\chapter{Experimental Results}
\inputx{Chapters}{4_results} % tex file in Chapters/
\chapter{Conclusions}
\inputx{Chapters}{5_conclusions}

\backmatter
\phantomsection
\addcontentsline{toc}{chapter}{References}
\bibliographystyle{Classes/jmb}
\bibliography{References/library}


\appendix
\inputx{Chapters}{appendices}

\end{document}
