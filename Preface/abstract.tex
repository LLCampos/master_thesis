\begin{abstracts}

Radiology reports describe the results of radiography procedures and have the potential of being an useful source of information which can bring benefits to health care systems around the world. One way to automatically extract information from the reports is by using Text Mining tools. The problem is that these tools are mostly developed for English and reports are usually written in the native language of the radiologist, which is not necessarily English. This creates an obstacle to the sharing of Radiology information between different communities.

This work explores the solution of translating the reports to English before applying the Text Mining tools, probing the question of what translation approach should be used. Having this goal, I created MRRAD (Multilingual Radiology Research Articles Dataset), a parallel corpus of Portuguese research articles related to Radiology and a number of alternative translations (human, automatic and semi-automatic) to English. This is a novel corpus which can be used to move forward the research on this topic. 

Using MRRAD I studied which kind of automatic or semi-automatic translation approach is more effective on the Named-entity recognition task of finding RadLex terms in the English version of the articles. Considering the terms extracted from human translations as our gold standard, I calculated how similar to this standard were the terms extracted using other translations. I found that a completely automatic translation approach using Google leads to F-Scores (between 0.861 and 0.868, depending on the extraction approach) similar to the ones obtained through a more expensive semi-automatic translation approach using Unbabel (between 0.862 and 0.870). To better understand the results I also performed a qualitative analysis of the type of errors found in the automatic and semi-automatic translations.

\end{abstracts}
