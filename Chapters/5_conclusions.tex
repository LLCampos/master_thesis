\label{chap5}

In the Introduction of this thesis I wrote that I was going to answer the question "lacking the resources to pay for human translation services, what kind of automatic (MT) or semi-automatic translation (MT+PE) approach should be used in the task of translating Portuguese Radiology-related text to English, for the purposes of finding RadLex terms in the translated text?". 

For this purposes, I have created the MRRAD corpus, a corpus of 51 Portuguese research articles related to radiology and four alternative translations to English for each one of these articles. This corpus can be used to study the efficacy of translation solutions in biomedical text, particularly text related to Radiology. To the best of my knowledge this is the first corpus of this type. This corpus could even be extended by other researchers, using different types of translation or languages, for example. 

Using this corpus I did a quantitative evaluation of the performance of multiple automatic or semi-automatic translation approaches in the task of translating Portuguese Radiology-related text to English, for the purposes of recognizing RadLex terms in the translated text. To better understand the results I also did an qualitative analysis of the type of errors found. The results will certainly be helpful for the decision-making of developers who want to develop multilingual applications that apply Text Mining tools, specially in Radiology text. The results corroborates the conclusion that if the developers have limited financial resources to pay for Human Translations, they will be better of using a Machine Translation service like Google instead of a service that implements Post-Editing, like Unbabel. Of course, maybe there are better Machine Translation services than Google or better Machine Translation + Post-Editing services than Unbabel is currently offering for the medical field, and this is something that could be explored in further work. 

Since this work explores a way to annotate non-English text using English terms, these results can motivate the sharing of annotations of biomedical text across communities. Linked-data \citep{Barros2016} approaches, for example, will benefit from this sharing because they will have access to data that would be hard to access behind language barriers. This sharing will allow, for example, find reports from different languages when searching for Radiology reports about left shoulders.

In this dissertation I just assessed the application of recognizing RadLex terms from translated text. A more realistic approach would be to test the performance of each kind of translation in a real application, like a Information Retrieval \citep{Manning2009b} or Question Answering system. But even if we discover which translation strategy is better for each kind of system, the question of the feasibility of integrating translation in systems used in real-word settings (e.g. hospitals) remains and this is something that could be explored in further work, through, for example, a partnership with a clinical facility.

Doing translations of Radiology reports to be consumed by software its just part of what needs to done to break language barriers in this field. Web platforms like Radiopaedia\footnote{\url{https://radiopaedia.org/}}, MyPACS\footnote{\url{https://www.mypacs.net}} and AuntMinnie\footnote{\url{http://www.auntminnie.com/}} have the goal of sharing radiological information in the Radiology community, but the information available is in English, which could be a obstacle to some Radiologists. Not just because of difficulties in writing or reading English, but the fact that the text is not in the native language of the user can make her feel less welcome to the community. My point being that further work could explore the task of translating Radiology text for human consumption.