\label{chap1}

\section{Motivation}
\label{motivation}

Radiology reports describe the results of radiography procedures (e.g., X-ray imaging) and have the potential of being an useful source of information, which can bring benefits to health care systems around the world. But because these reports are written in a free-text mode, it is hard to extract information automatically from them. This was more problematic when these reports were mostly stored in physical format (paper) - the fact that these reports can now be accessed digitally make them amenable for processing using Text Mining techniques. 

A lot of work has been done on this research topic \citep{Pons2016}, but it is usually assumed that the reports are written in English. For example, \citep{Hassanpour2016} created an information extraction system for English reports that depends on RadLex, a lexicon for radiography terminology, which is freely available in English. Because of this, the system can not be applied to reports written in other languages. And even if the system was not dependable on an English lexicon, it is not certain if the results would be the same if text in another language was used.

Assuming that the information automatically extracted from radiology reports using Text Mining techniques can bring benefits to health care systems, this waste of information caused by language barriers can potentially have a negative impact on everyone's health.

If what is wanted is to use tools that depend on English lexicons, one possible solution to the problem could be to translate the lexicon itself \citep{Bretschneider}. Other obvious solution, and the one explored in this thesis, is to translate the reports. If the translation is done by professionals trained in the translation of medical texts, we probably can assume that not much information is lost in translation. We call this type of translation Human Translation (HT). But professional translators are expensive and there are a lot of reports to translate, so this would have a really high monetary cost. Another option is to use Machine Translation (MT). Notwithstanding the lower translation quality, it is way cheaper and more feasible in a large scale. Finally, an option that tries to get the best of both worlds is using Machine Translation with Post-Editing (MT-PE) by humans. Basically, the text is automatically translated by a machine and then the translation is corrected by a human. Cheaper than the HT option and with better quality than the MT one. 

So, how much information is modified or lost during MT or MT+PE compared to HT, affecting the results of Text Mining tools? To the best of my knowledge, the most similar work to this one is \citep{Castilla2007a}. He founds that a rule-based MT system has a good performance in translating Portuguese text to English for the purposes of applying a text mining tool (better described in \ref{multilingual-text-mining}). The author doesn't compare translation systems, something that is done on the present work.

\section{Objectives}

In this thesis I have studied how MT and MT+PE compares with HT on the simple task of Named-entity recognition (NER) using a gazetteer-based approach, this gazetteer consisting of RadLex terms. The identification of RadLex terms can be useful, for example, in image retrieval \citep{Gerstmair2012} systems, so this is not just a toy example. 

If I have non-English radiology reports and I want to translate them so that I can identify RadLex terms for use on some other system, what kind of translation should I use? In this thesis, I try to help to answer this question. 

\newcommand{\hypothesis}{
\begin{description}
	\item[Hypothesis:] MT+PE is a good trade-off between quality and cost, compared with MT and HT, for translating radiology reports for the purpose of identifying RadLex terms. 
\end{description}
}
\hypothesis

For this to be true, these conditions have to hold:

\begin{enumerate}
	\item MT+PE has to be cheaper than HT
	\item MT+PE quality for the task at hand has to be close enough to HT quality
	\item MT+PE quality for the task at hand has to be better than MT quality, enough to compensate its higher cost
\end{enumerate}

The last condition its important because if MT+PE quality is similar to MT quality, as MT cost is lower, maybe it's worth to just use MT. In this thesis I only try to answer to the quality issues, not doing a thorough economic analysis of the problem. 

\section{Methodology}

To answer these questions I've compared the results of the NER task on the MT and MT+PE translations with the results of the NER task on the HT translation. Assuming that the HT translation is the right translation, the closer the results are to the ones of the HT, the better the translation for the task at hand. So, for the proposed hypothesis to be true, the results of the NER task on the MT+PE task have to be closer to the results of the HT translation than the results of the MT translations.

For this purposes I've created a parallel corpus containing a number of scientific articles related to radiology and corresponding HT, MT and MT+PE translations. These reports were annotated with RadLex terms using the Open Biomedical Annotator and NOBLE Coder. The annotations of the MT and MT+PE translations were then compared with the ones from HT.


\section{Contributions}

This thesis lead to the following contributions: 



\begin{itemize}

\item \textbf{MRRAD Corpus}
	\begin{itemize}
		\item A Portuguese-English parallel corpus of research articles related to Radiology, called MRRAD (Multilingual Radiology Research Articles Dataset), containing for each article the original Portuguese document,the HT translation, two alternative MT translations and a MT+PE translation;
	\end{itemize}
	
\item \textbf{Main Scientific Results}
	\begin{itemize}
		\item Measurement of the performance of multiple automatic or semi-automatic translation approaches in the task of translating Portuguese Radiology-related text to English, for the purposes of finding RadLex terms in the translated text;
	\end{itemize}

\item \textbf{Bioinformatics Open Days 2017}\footnote{\url{http://bioinformaticsopendays.com/}}
	\begin{itemize}
	\item Abstract submission and oral presentation describing this work;
	\item Organization and presentation of workshop on Biomedical Text Mining with other members of the LaSIGE team\footnote{\url{https://sites.google.com/view/biomedicaltextminingworkshop}};
	\end{itemize}
	
\item \textbf{Scientific Publications}
	\begin{itemize}
		\item ...
	\end{itemize}
	
\item \textbf{Other Open-Source Contribuitions}
	\begin{itemize}
		\item A Python binding of the BioPortal REST API\footnote{\url{https://github.com/LLCampos/pybioportal}};
		\item Converter of NOBLE Coder annotation file to Webanno TSV 2 annotations files\footnote{\url{https://gist.github.com/LLCampos/5f1680941984c4b63f986965e7384e6c}};
	\end{itemize}
	
\item \textbf{Multilingual Report Annotator}
	\begin{itemize}
		\item Development of a proof of concept web application for translation and annotation of Radiology text;\footnote{\url{http://www.lasige.di.fc.ul.pt/webtools/mra/}}
	\end{itemize}
	 
\end{itemize}


