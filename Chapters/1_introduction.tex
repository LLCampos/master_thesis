\label{chap1}

\section{Motivation}
\label{motivation}

Radiology reports describe the results of radiography procedures and have the potential of being an useful source of information, which can bring benefits to health care systems around the world. However, these reports are usually written in free-text and thus it is hard to automatically extract information from them. Nonetheless, the fact that most reports are now digitally available make them amenable for using Text Mining tools. Another advantage of Radiology reports is that even if written in free-text, they are usually well structured.

A lot of work has been done on Text Mining of Biomedical texts, including health records \citep{Pons2016}, but although Radiology reports are usually written in the native language of the radiologist, Text Mining tools are mostly developed for English. For example, \citep{Hassanpour2016} created an information extraction system for English reports that depends on RadLex, a lexicon for radiography terminology, which is freely available in English. Given this dependence, the system cannot be easily applied to reports written in other languages. And even if the system was not dependable on an English lexicon, it is not certain that the results would be the same if another language was used, because of, for example, differences in syntax. This have been an obstacle in the sharing of Radiology information between different communities, which is important to understand and effectively address health problems.

There are mainly two possible solutions to this problem. One is to translate the lexicon itself \citep{Bretschneider} and the other is to translate the reports. Translating the lexicon has the advantage of not requiring continuous translation, i.e., after translating a lexicon to, for example, Spanish, we can then use it to process as many untranslated Spanish reports as needed. However, when a new version of the lexicon is released the changes need also to be translated, otherwise the translated lexicon would become outdated. Given the increasing evolution of translation services nowadays available, in this work I assess the alternative option of translating the reports and check its feasibility. This approach has the advantage that the translated reports would be accessible to any doctor who understands English and any state-of-the-art Text Mining tools focused on English text can be applied without any need for adaptation. 

If the translation is done by professionals trained in the translation of medical texts, we probably can assume that not much information is lost in translation. We call this type of translation Human Translation (HT). But expert translators are expensive, which makes this solution unscalable, with a high financial cost. Another option is to use Machine Translation (MT). Notwithstanding the lower translation quality, it is way cheaper and more feasible in a large scale. Finally, an option that tries to get the best of both worlds is using Machine Translation with Post-Editing (MT-PE) by humans. In this approach the text is automatically translated and then the translation is corrected by a human. Cheaper than the HT option and with better quality than the MT one.

The choice of translation approach its important because it will affect the quality of the output of the Text Mining tools. To the best of my knowledge, currently there is no publicly available study that provided a quantitative evidence that would help make this choice. This could be explained by the lack of a parallel corpus that could be used to study this. To the best of my knowledge, the most similar work to this one is \citep{Castilla2007a}. He founds that a rule-based MT system has a good performance in translating Portuguese text to English for the purposes of applying a text mining tool (better described in \ref{multilingual-text-mining}). The author does not compare translation systems, something that is done on the present work.

Specifically, I focused on the Text Mining task of Named-entity recognition (NER). This is a relevant task since the outputs from NER systems can be used in Image Retrieval \citep{Gerstmair2012} and Information Retrieval \citep{Antony2015} systems and can be useful for improving automatic Question Answering \citep{Toral2005}.

\section{Objectives}

Thus, I aimed at addressing the following research question: lacking the resources to pay for Human Translation services, what kind of automatic (MT) or semi-automatic translation (MT+PE) approach should be used in the task of translating Portuguese Radiology-related text to English, for the purposes of finding RadLex terms in the translated text? I propose the following hypothesis:

\newcommand{\hypothesis}{
\begin{description}
	\item[Hypothesis:] MT+PE is a good trade-off between quality and cost, compared with MT and HT, for translating Portuguese Radiology reports to English, for the purpose of identifying RadLex terms in the translated text. 
\end{description}
}
\hypothesis

For this to be true, these conditions have to hold:

\begin{enumerate}
	\item MT+PE has to be cheaper than HT
	\item The terms identified in the MT+PE translations have to be similar enough to the ones identified in the HT translation
	\item The terms identified in MT+PE translations have to be more similar to the ones identified in the HT translation than the ones identified in MT translations
\end{enumerate}

The first condition is known to be true. The last condition its important because if MT+PE quality is similar to MT quality, as MT cost is lower, then it is worth to just use MT. In this thesis I only try to answer to the quality issues, not doing a thorough economic analysis of the problem. 

\section{Methodology}

To test this hypothesis I have compared the RadLex terms identified in MT and MT+PE translations to the terms identified in HT translations, which I assumed to be a gold standard.

For this purposes I've created MRRAD, a parallel corpus containing 51 Portuguese scientific articles related to Radiology and corresponding HT, MT (Google and Yandex) and MT+PE (Unbabel) English translations. These translations were annotated with RadLex terms using the Open Biomedical Annotator and NOBLE Coder. More than one annotation approach was used to experiment with different kinds annotation approaches. For each translation and annotation approach I created the set of the RadLex terms that were found in that translations with that annotation approach. The terms found in the MT and MT+PE translations were then compared with the ones found in the HT translations.

The MRRAD corpus and annotations used in this work can be found in a public GitHub repository\footnote{\url{https://github.com/lasigeBioTM/MRRAD}}.


\section{Contributions}

This thesis lead to the following contributions: 

\begin{itemize}

\item \textbf{MRRAD Corpus}
	\begin{itemize}
		\item A Portuguese-English parallel corpus of research articles related to Radiology, called MRRAD (Multilingual Radiology Research Articles Dataset), containing for each article the original Portuguese document, the HT translation, two alternative MT translations and a MT+PE translation;
	\end{itemize}
	
\item \textbf{Main Scientific Results}
	\begin{itemize}
		\item Measurement of the performance of multiple automatic or semi-automatic translation approaches in the task of translating Portuguese Radiology-related text to English, for the purposes of finding RadLex terms in the translated text;
	\end{itemize}

\item \textbf{Bioinformatics Open Days 2017}\footnote{\url{http://bioinformaticsopendays.com/}}
	\begin{itemize}
	\item Abstract submission and oral presentation describing this work;
	\item Organization and presentation of workshop on Biomedical Text Mining with other members of the LaSIGE team\footnote{\url{https://sites.google.com/view/biomedicaltextminingworkshop}};
	\end{itemize}
	
\item \textbf{Scientific Publications}
	\begin{itemize}
		\item ...
	\end{itemize}
	
\item \textbf{Other Open-Source Contribuitions}
	\begin{itemize}
		\item A Python binding of the BioPortal REST API\footnote{\url{https://github.com/LLCampos/pybioportal}};
		\item Converter of NOBLE Coder annotation file to Webanno TSV 2 annotations files\footnote{\url{https://gist.github.com/LLCampos/5f1680941984c4b63f986965e7384e6c}};
	\end{itemize}
	
\item \textbf{Multilingual Report Annotator}
	\begin{itemize}
		\item Development of a proof of concept web application for translation and annotation of Radiology text \citep{Campos2017};
	\end{itemize}
	 
\end{itemize}


